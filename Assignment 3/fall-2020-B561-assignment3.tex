\documentclass{article}
\usepackage{amsmath}
\usepackage{enumitem}

\begin{document}

\centerline{\Large B561 Advanced Database Concepts}
\medskip
\centerline{\Large Assignment 3}

\medskip
\medskip

This assignment is designed to test your knowledge of the following lectures:
\begin{itemize}
\item Lecture 6:  Aggregate functions and data partitioning
\item Lecture 7:  Queries with quantifiers (Part 1)
\item Lecture 8:  Queries with quantifiers (Part 2)
\item Lecture 9:  Triggers
\end{itemize}

To turn in your assignment, you will need to upload to Canvas a single file with name {\tt assignment3.sql} which contains 
the necessary SQL statements that solve the problems in this assignment.   
The {\tt assignment3.sql} file must be such that the AI's can run it in their PostgreSQL environment.  
In addition, you will need to upload a separate {\tt assignment3.txt} file that contains the results of running
your queries.
We have posted the exact requirements and an example for uploading your solution files.  (See the module
{\tt Instructions for turning in assignments}.)

For most of the problems in the assignment, we will use a database
that maintains a set of persons {\tt Person}, a relation {\tt Knows}, a set of companies {\tt Company}, a relation {\tt WorksFor}, 
a set of job skills ({\tt JobSkill}), and a relation {\tt PersonSkills}.   
(The data for this database are given along with this assignment 
in the {\tt data.sql} file.)
The schemas for these sets and relations are as follows (primary keys are underlined):

{\small
\begin{center}
{\tt
  \begin{tabular}{l}
  {Person}($\underline{\tt pid: integer}$, name:text, city: text, birthYear: integer)\\
  {Knows}($\underline{\tt pid1:integer, pid2:integer}$) \\
  {Company}($\underline{\tt cname:text,  city:text}$) \\
  {WorksFor}($\underline{\tt pid: integer}$, cname:text, salary:integer) \\
   {JobSkill}($\underline{\tt skill:text }$) \\
  {PersonSkill}($\underline{\tt pid: text, skill:text}$),\\
  \end{tabular}
  }
\end{center}
}

\begin{itemize}
\item The {\tt city} and {\tt birthYear} in {\tt Person}
specify the city in which the person lives and his or her birth year.

\item The relation {\tt Knows} maintains a set of pairs $(p_1,p_2)$ where $p_1$ 
and $p_2$ are pids of persons.   The pair $(p_1,p_2)$ indicates that the person with
pid $p_1$ knows the person with pid $p_2$.
We do not assume that the relation {\tt Knows} is
symmetric: it is possible that $(p_1,p_2)$ is in the relation but that
$(p_2,p_1)$ is not.

\item The {\tt city} attribute in
{\tt Company} indicates a city in which the company is located.
(Companies may be located in multiple cities.)

\item The relation {\tt WorksFor} stores the unique company (identified by {\tt cname}) for which a person
works along with the salary he or she makes at that company.
(Incidentally, it is possible that a person in the {\tt Person} relation does not
work for any company.)

\item The relation {\tt JobSkill} only has the attribute {\tt skill} which is the name of a possible
job skill.    

\item The relation {\tt PersonSkill} provides for each person his or her job skills.
A person may have multiple job skills.   It is also possible that a person does not
have any job skills.
\end{itemize}

We assume the following primary key and foreign key constraints:
\begin{itemize}
\item {\tt pid} is the primary key of {\tt Person}
\item ({\tt pid1}, {\tt pid2}) is the primary key of {\tt Knows}
\item ({\tt cname}, {\tt city}) is the primary key of {\tt Company}
\item {\tt pid} is the primary key of {\tt WorksFor}
\item {\tt skill} is the primary key of {\tt JobSkill}
\item ({\tt pid}, {\tt skill}) is the primary key of {\tt PersonSkill}
\item {\tt pid1} is a foreign key in {\tt Knows} referencing the primary key {\tt pid} in {\tt Person}
\item {\tt pid2} is a foreign key in {\tt Knows} referencing the primary key {\tt pid} in {\tt Person}
\item {\tt pid} is a foreign key in {\tt WorksFor} referencing the primary key {\tt pid} in {\tt Person}
\item {\tt cname} in {\tt WorksFor} references a {\tt cname} that appears in {\tt Company}
\item {\tt pid} is a foreign key in {\tt PersonSkill} referencing the primary key {\tt pid} in {\tt Person}
\item {\tt skill} is a foreign key in {\tt PersonSkill} referencing the primary key {\tt skill} in {\tt Skill}
\end{itemize}




\newpage
\section{Operations on polynomials and matrices using SQL aggregate functions}

In the problems in this section, you will practice working with aggregate functions.

A useful other aspect of solving these problems is that you will learn how relations can be used
to represent polynomials and matrices and how SQL can be used to define operations on
such objects.


\begin{enumerate}

\item Let $P(x)$ be a polynomial with integer coefficients.   For example, $P(x)$ could be the polynomial
$3x^3 -2x^2 + 5$.
  
We can represent a polynomial $P(x)$ with a binary relation
\[\text{{\bf P}(coefficient: integer, degree: integer)}\]
wherein each pair $(c,d)$ represents the term
$cx^d$ in $P(x)$.    For example, $P(x) = 3x^3 -2x^2 + 5$ is represented in {\bf P} as
follows:

\begin{center}
\begin{tabular}{c}
{\bf P} \\ 
 $
\begin{array}{c|c}
\mbox{\rm coefficient} & \mbox{\rm degree} \\ \hline
3 & 3 \\
-2 & 2 \\
0 & 1\\
5 & 0 \\
\end{array}$
\end{tabular}
\end{center}


Write a SQL function

\begin{center}
{\tt 
\begin{tabular}{l}
create or replace function evaluatePolynomial(x numeric)\\
$\qquad\quad$ returns numeric as \\
\$\$ \\
... \\
\$\$ language sql
\end{tabular}
}
\end{center}

\medskip
such that, for an input number $x_0$,

\begin{center}
\medskip
{\tt 
\begin{tabular}{l}
select evaluatePolynomial($x_0$);
\end{tabular}
}
\end{center}

returns the value $P(x_0)$. For example,
for the polynomial $P(x) =  3x^3 -2x^2 + 5$ and $x_0=7$,
$P(7) = 3\times7^3 -2\times7^2 + 5 = 936$ and so

\begin{center}
\medskip
{\tt 
\begin{tabular}{l}
select evaluatePolynomial(7);
\end{tabular}
}
\end{center}

should return the value 936.

Your solution should work for any polynomial $P(x)$ of degree $n$, for $n\geq 0$.
Observe that if $P(x) = c_0 + c_1x +
c_2x^2+\cdots+c_nx^n$ is such a polynomial, then $P(x_0)$ is an aggregated sum, i.e.,
$P(x_0) = \sum_{k=0}^n\, c_d(x_0)^k$.   (You can use the PostgreSQL function
$\text{\tt power}(x,k)$ to computes the value $x^k$.)

\item Let $p(x)$ and $q(x)$ be 2 polynomials with integer coefficients.  

Let 
{\tt P(coefficient, degree)} and
{\tt Q(coefficient, degree)} be two binary relations representing $p(x)$ and $q(x)$, respectively.  
E.g., if $p(x) = 2x^2-5x+5$ and $q(x) = 4x^4+ 3x^3 + x^2 - x$  
then their representations in the relations {\bf P} and {\bf Q} are as follows:

\begin{center}
\begin{tabular}{ccc}
\begin{tabular}{c}
{\bf P} \\ 
 $
\begin{array}{c|c}
\mbox{\rm coefficient} & \mbox{\rm degree} \\ \hline
2 & 2 \\
-5 & 1\\
5 & 0 \\
\end{array}$
\end{tabular}
&
\begin{tabular}{c}
{\bf Q} \\ 
 $
\begin{array}{c|c}
\mbox{\rm coefficient} & \mbox{\rm degree} \\ \hline
4 & 4\\
3 & 3 \\
1 & 2 \\
-1 & 1 \\
0  & 0
\end{array}$
\end{tabular}
\end{tabular}


\end{center}

Write a SQL function
\begin{center}
{\tt 
\begin{tabular}{l}
create or replace function multiplicationPandQ() \\
$\qquad\quad$ returns table(coefficient integer, degree integer) as \\
\$\$ \\
... \\
\$\$ language sql;
\end{tabular}
}
\end{center}
that computes a binary relation representing the 
multiplication of $p(x)$ and $q(x)$, i.e., the polynomial $p(x) * q(x)$.  

For example, consider $p(x) = 2x^2 -5 x +5$ and $q(x) = 4x^4+ 3x^3 + x^2 - x$.
Then $p(x)*q(x) = (8)x^6 + (6-20) x^5 +(2-15+20)x^4 +(-2-5+15)x^3+(5+5)x^2+(-5)x=8x^6-14x^5+7x^4+8x^3+10x^2-5x$.
So, for these polynomials, your SQL query should return the relation

\begin{center}
\begin{tabular}{c}
 $
\begin{array}{c|c}
\mbox{\rm coefficient} & \mbox{\rm degree} \\ \hline
8 & 6 \\
-14 & 5 \\
7 & 4 \\
8 & 3 \\
10 & 2\\
- 5 & 1\\
0 & 0
\end{array}$
\end{tabular}
\end{center}



Your solution should work for arbitrary polynomials $p(x)$ and $q(x)$.
 

\item Let $X=(x_1,\ldots,x_n)$ and $Y=(y_1,\ldots,y_n)$ be two $n$-dimensional
  vectors of numbers. (We will assume that $n\geq 1$).   For example, for $n=3$, $X$ could be the vector $(7,-1,2)$ and $Y$ the vector $(1,1,-10)$.

  The \emph{dot product} $X\cdot Y$ of $X$ and $Y$ is defined as the aggregated sum
  \[x_1\times y_1+x_2\times y_2+\cdots+x_n\times y_n=\sum_{i=1}^n\, x_i\times y_i\]

We will represent the vector $X$ with a binary relation
{\bf X}(index,value) such that $(i,v)$ is in {\bf X} if $x_i = v$.  
Analogously, we will represent the vector $Y$ with a binary relation {\bf Y}(index,value).
For example, for the vectors $X=(7,-1,2)$ and $Y=(1,1,-10)$,
{\bf X} and {\bf Y} are the following binary relations:

\begin{center}
  \begin{tabular}{ccc}
\begin{tabular}{c}
{\bf X} \\ 
 $
\begin{array}{c|c}
\mbox{\rm index} & \mbox{\rm value} \\ \hline
1 & 7 \\
2 & -1 \\
3 & 2 \\
\end{array}$
\end{tabular}
    & \hskip 1in &
\begin{tabular}{c}
{\bf Y} \\ 
 $
\begin{array}{c|c}
\mbox{\rm index} & \mbox{\rm value} \\ \hline
1 & 1 \\
2 & 1 \\
3 & -10 \\
\end{array}$
\end{tabular}
  \end{tabular}
  \end{center}

Write a SQL function

\medskip
\begin{center}
{\tt 
\begin{tabular}{l}
create or replace function dotProductXandY() returns numeric as \\
\$\$ \\
... \\
\$\$ language sql;
\end{tabular}
}
\end{center}

\medskip
such that, {\tt select dotProductXandY()}
returns the value $X\cdot Y$. For example,
for the vectors $X = (7,-1,2)$ and $Y = (1,1,-10)$,
$X\cdot Y = 7\times 1 + (-1)\times 1 + 2\times (-10) = 7 - 1-20=-14$, and therefore
{\tt select dotProductXandY()}
should return the value -14.

Your solution should work for any pair of $n$-dimensional
vectors $X$ and $Y$, with $n\geq 1$.
\end{enumerate}


\newpage
\begin{enumerate}[resume]
\item  Let $M$ be an $n\times n$ matrix of integers with $n\geq 1$.

For $i, j\in [1,n]$, we will denote by $M[i,j]$ the
element in matrix $M$ at row $i$ and column $j$.   We will assume that $n\geq 1$.

Given two $n\times n$ matrix $M$ and $N$, denote by $M * N$ the matrix multiplication of $M$ and $N$.  
By definition, $M * N$ is again a $n\times n$ matrix where, 
for $i, j\in [1,n]$, $M* N[i,j]$ is defined by
{\small
\[M * N[i,j] = \sum_{k=1}^n M[i,k]\times N[k,j].\]
}

The matrix $M$ can be represented using a relation {\tt M} with schema 
\begin{center}\begin{tabular}{c}
({\tt row: integer}, {\tt colmn: integer}, {\tt value: integer})\end{tabular}\end{center} and similarly for
the matrix $N$.\footnote{Notice that we use the attribute name `{\tt colmn}' since the word `{\tt column}' is a reserved
word in PostgreSQL.}

For example if $M$ is the $3\times 3$ matrix
\[
M = \begin{matrix}
         1 & 2 & 3 \\
         1 & -3 & 5 \\
         4 & 0 &- 2
         \end{matrix}
\]

then ${\tt M}$ is the following relation of $9$ tuples:

\[
\begin{array}{c}
{\tt M} \\ 
 \begin{array}{ccc}
 {\tt row} & {\tt colmn} & {\tt value} \\ \hline
1 & 1 & 1 \\
1 & 2 & 2 \\
1 & 3 & 3 \\
2 & 1 & 1 \\
2 & 2 & -3 \\
2 & 3 & 5 \\
3 & 1 & 4 \\
3 & 2 & 0\\
3 & 3 & -2 \\
 \end{array}
\end{array}
\]

Let $M$ and $N$ be two $n\times n$ matrices represented by the two relations {\tt M} and {\tt N}.

Write a SQL query that computes a relation over schema ({\tt row}, {\tt column}, {\tt value}) that represents
the matrix $M * N$. 
Your solution should work for any $n\geq 1$.

For example if $M$ and $N$ are given by the following $3\times 3$ matrices stored as the relations

\[
\begin{array}{ccc}
\begin{array}{c}
{\tt M} \\ 
 \begin{array}{ccc}
 {\tt row} & {\tt colmn} & {\tt value} \\ \hline
1 & 1 & 1 \\
1 & 2 & 2 \\
1 & 3 & 3 \\
2 & 1 & 1 \\
2 & 2 & -3 \\
2 & 3 & 5 \\
3 & 1 & 4 \\
3 & 2 & 0\\
3 & 3 & -2 \\
 \end{array}
\end{array}
&
\begin{array}{c}
{\tt N} \\ 
 \begin{array}{ccc}
 {\tt row} & {\tt colmn} & {\tt value} \\ \hline
1 & 1 & -1 \\
1 & 2 & 2 \\
1 & 3 & -1 \\
2 & 1 & 2\\
2 & 2 & -3 \\
2 & 3 & 4 \\
3 & 1 & 0 \\
3 & 2 & 0\\
3 & 3 & 3 \\
 \end{array}
\end{array}
\end{array}
\]
%
then your query should produce the relational representation of $M * N$, i.e., the relation

\[
\begin{array}{c}
{\tt M*N} \\ 
 \begin{array}{ccc}
 {\tt row} & {\tt colmn} & {\tt value} \\ \hline
1 & 1 & 3 \\
1 & 2 & -4 \\
1 & 3 & 16 \\
2 & 1 & -7\\
2 & 2 & 11 \\
2 & 3 & 2 \\
3 & 1 & -4 \\
3 & 2 & 8\\
3 & 3 & -10 \\
 \end{array}
\end{array}
\]



\end{enumerate}

\newpage
\section{Simulating Set Predicates with the COUNT Function}

The problems in this section aim to illustrate that the set predicates of SQL are redundant provided one can use the
COUNT aggregate function.   This underscores the importance and naturalness of counting to express quantifiers.

Rewrite each of the following SQL queries into an equivalent SQL query wherein the
set predicates  are simulated using the {\tt COUNT} aggregate function.
In other words, you solution should use the COUNT aggregate function and can not use any of the [NOT] EXISTS, [NOT] IN, 
SOME, and ALL set predicates.   You can also not use the set operations UNION, INTERSECT, and EXCEPT.

\begin{enumerate}[resume]
\item  
{\small
\begin{verbatim}
select p.pid, p.name
from   Person p
where  p.city = 'Bloomington' and 
       exists(select 1
              from   personSkill ps
              where  ps.pid = p.pid and ps.skill  <> 'Programming' and
                     ps.skill in (select ps1.skill
                                  from   worksFor w1, personSkill ps1
                                  where  w1.cname = 'Netflix' and 
                                         ps1.skill <> 'AI' and 
                                         w1.pid = p.pid));
\end{verbatim}}
\item 
{\small
\begin{verbatim}
select w.pid, w.cname, w.salary
from   worksFor w
where  not(w.salary <= all (select w1.salary
                            from   worksFor w1, Company c
                            where  w.pid <> w1.pid and w.cname = w1.cname and 
                                   c.city not in (select p.city
                                                  from   Person p
                                                  where  p.birthyear <> 1990)));
                                               
\end{verbatim}
}
\end{enumerate}
\newpage
\section{Solving queries using aggregate functions}

Formulate the following queries in SQL.   You should use aggregate functions to solve these queries.  You can use views,
including temporary views as well as parameterized views defined by
user-defined functions that return relations (i.e., tables).

\begin{enumerate}[resume]
\item Find the pid and name of each person who lives in `Bloomington' and who knows at most one
person who has at least 3 job skills.
\item Find the pid and name of each person who has all but three job skills.   I.e., such a person lacks precisely three
job skills from the possible job skills that are stored in the relation {\tt jobSkill}.    
\item Find the cname of each company along with the number of persons who work for that company and
who know at least two other persons who also work for that company.
(Make sure to also consider companies that have no such employees and to include for each
such company a tuple of the form (cname,0) in the answer.)
\item  Find the pid and name of each person who works for 'Netflix' and who knows the most persons who work for `IBM'.
\item Find the cname of each company who spends the highest aggregated amount on the salaries of persons who 
work for that company and who have the fewest skills.
\end{enumerate}
\newpage
\section{Queries with quantifiers using Venn diagrams with conditions}

Using the method of Venn diagrams with conditions and without using the {\tt COUNT} function,
write SQL queries for the following queries with quantifiers.

In these problems, you must write appropriate views and parameterized views for the sets $A$ and $B$
that occur in the Venn diagram with conditions for these queries.  (See the lecture on Queries with Quantifiers.)

\begin{enumerate}[resume]
\item Find the pid and name of each person who does not know any person who has a salary strictly above 55000.
\item Find the pid and name of each person who knows all the persons who (a) work  at Netflix, (b) make at least 55000, and
(c) are born after 1985.

\noindent
{\bf Changed 'are born before 1990' by 'are born after 1985'}.

\item Find the cname of each company who only employs persons who make less than 55000.
\item Find the pairs of different skills $(s_1,s_2)$ such that not every person who works at `IBM' and has skill $s_1$  
knows a person who works at `Netflix' and has skill $s_2$.
\item The concept of Venn diagrams with conditions can be generalized to more than 2 sets.   Here is a query that requires
a condition on a Venn diagram with 3 sets.   To solve this problem, you should consider definining 3 parameterized views.

Find each triple $(p, c, s)$ where $p$ is the pid of a person, $c$ is the cname of a company, and $s$ is a skill, such that
each person who is known by the person with pid $p$ and who works for company with cname $c$ is a person who has the jobskill $s$.


\end{enumerate}

\newpage
\section{Queries with quantifiers using Venn diagrams with counting conditions}

Using the method of Venn diagram with counting conditions, write SQL queries for 
the following queries with quantifiers.

In these problems, you should write appropriate views and parameterized views for the sets $A$ and $B$
that occur in the Venn diagrams for these queries.  (See the lecture on Queries with Quantifiers Using the COUNT function.)

\begin{enumerate}[resume]
\item Find the pid and name of each person who has more than 2 of the combined set of job skills of
persons who work for `IBM'.

{\bf Changed `more than 3' to more than 2' and `Netflix' to 'IBM'}.

\item Find the cname of each company that employs an odd number of persons whose salary is at least 50000.
\item Find the pid and name of each person who knows at least 2 persons who each have exactly 3 job skills.
\item Find the pairs $(p_1,p_2)$ of different person pids such that the person with pid $p_1$ and the
person with pid $p_2$ knows the same set of persons.
\item Find the pairs $(p_1,p_2)$ of different person pids such that the person with pid $p_1$ and
the person with pid $p_2$ knows the same number of persons.
\end{enumerate}
\newpage
\end{document}
\section{Triggers}

To begin the problems in this section, you should first remove the entire database, including the relation schemas.
You should then create the relations but without specifying the primary and foreign key constraints.
You should also not yet populate the relations with data.

Solve the following problems:
\begin{enumerate}[resume]
\item \label{firstTriggerQuestion} Develop appropriate insert and delete triggers that implement the primary key and foreign key constraints
that are specified for the {\tt Person}, {\tt Knows}, {\tt Company}, and {\tt Worksfor} relations.
Your triggers should report appropriate error conditions.

For this problem, implement the triggers such that foreign key constraints are maintained
using the cascading delete semantics. 

For a reference on cascading deletes associated with foreign keys maintenance consult the PostgreSQL manual page
\begin{center}
{\tt https:\/\/www.postgresql.org\/docs\/9.2\/ddl-constraints.html}
\end{center}

Test your triggers using appropriate inserts and deletes.

\item Repeat Problem~\ref{firstTriggerQuestion} but now implement the triggers such that foreign key constraints are maintained
using the restrict delete semantics.

For a reference on restricted deletes associated with foreign keys maintenance consult the PostgreSQL manual page
\begin{center}
{\tt https:\/\/www.postgresql.org\/docs\/9.2\/ddl-constraints.html}
\end{center}

Test your triggers using appropriate inserts and deletes.

%\item Develop a trigger for the constraint that states that each {\tt cname} in {\tt Worksfor} must references  {\tt cname}
%in {\tt Company}.   Since the trigger resembles a foreign key constraints, use the cascade delete semantics for
%its implementation.
%
%Test your trigger using appropriate inserts and deletes.

\item Consider the following view definition

\begin{verbatim}
create or replace view PersonKnowsNumberofPersons as 
  (select p1.pid, p1.name, 
         (select count(1)
          from   Person p2
          where (p1.pid, p2.pid) in (select k.pid1, k.pid2
                                     from Knows k)) as NumberofKnownPersons
    from Person p1);                                                          
\end{verbatim}

This view defines the set of tuples $(p,n,c)$ where $p$ and $n$ are the pid and name of a person and $c$ is
the number of persons who are known by the person with pid $p$.

You should not create this view!
Rather your task is to create a relation {\tt  PersonKnowsNumberofPersons} that maintains a
materialized view in accordance with the above view definition under insert and delete operations on the
{\tt Person} and {\tt Knows} relation.

Your triggers should be designed to be incremental.   In particular, when an insert or delete occurs, you should not always
have to reevaluate all the number of persons known by persons.
Rather the maintenance of {\tt PersonKnowsNumberofPersons}
should only apply to the person information that is affected by the insert or delete.


\item Consider the following view definition

\begin{verbatim}
create or replace view BloomingtonPersonWhoKnowsAtLeastTwoPersons as 
  (select p.pid, p.name, p.birthyear
   from   Person p
   where  p.city = 'Bloomington' and
          exists (select 1
                  from   knows k1, knows k2
                  where  p.pid = k1.pid1 and
                         p.pid = k2.pid1 and
                         k1.pid2 <> k2.pid2));
                                     
\end{verbatim}

You should not create this view!
Rather your task is to create a relation {\tt  BloomingtonPersonWhoKnowsAtLeastTwoPersons} that maintains a
materialized view in accordance with the above view definition under insert and delete operations on the
{\tt Person} and {\tt Knows} relation.

In other words, the state of this relation needs to be maintained under inserts and deletes in the database
and should always reflect the definition of the view above.
Besides the relation  {\tt  BloomingtonPersonWhoKnowsAtLeastTwoPersons}, you are allowed to maintain
another relation (or other relations) that can assist in maintaining this materialized view.  For example,
you may wish to maintain a relation that keeps track of  the number of persons know for each employee.

Provide tests that show that your solution works.

%\item As currently used, the {\tt Knows} relation is not necessarily symmetric.  I.e., it is possible that
%$(p_1,p_2)$ is in {\tt Knows} but $(p_2,p_1)$ is not.
%
%In this problem, you are required to maintain a symmetric {\tt Knows} relation.  In other words, it must by the
%case that $(p_1,p_2)$ is in {\tt Knows} if and only if $(p_2,p_1)$ is in {\tt Knows}.
%
%Write insert and delete triggers that maintain the symmetricity requirement for the {\tt Knows} relation.  I.e.,
%if a pair $(p_1,p_2)$ is inserted in {\tt Knows} then a side-effect of an insert trigger should be that then both
%$(p_1,p_2)$ and $(p_2,p_1)$ become part of {\tt Knows}.    Furthermore, if $(p_1,p_2)$ is deleted from the {\tt Knows} relation then
%$(p_2,p_1)$ should also be deleted from {\tt Knows}.    
%
%To solve this problem, you can use auxiliary relations and or views to maintain the symmetricity condition of the
%{\tt Knows} relation.

\end{enumerate}







\end{document}




